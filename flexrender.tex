
% Cal Poly Thesis
% 
% based on UC Thesis format
%
% modified by Mark Barry 2/07.
%




\documentclass[12pt]{ucthesis}

%\newif\ifpdf
%\ifx\pdfoutput\undefined
%    \pdffalse % we are not running PDFLaTeX
%\else
%\pdfoutput=1 % we are running PDFLaTeX
%\pdftrue \fi

\usepackage{textcomp}
\usepackage{url}
\usepackage{listings}
\lstset{
	language=[Visual]C++,
	keywordstyle=\bfseries\ttfamily\color[rgb]{0,0,1},
	identifierstyle=\ttfamily,
	commentstyle=\color[rgb]{0.133,0.545,0.133},
	stringstyle=\ttfamily\color[rgb]{0.627,0.126,0.941},
	showstringspaces=false,
	basicstyle=\small,
	numberstyle=\footnotesize,
	numbers=left,
	stepnumber=1,
	numbersep=10pt,
	tabsize=2,
	breaklines=true,
	prebreak = \raisebox{0ex}[0ex][0ex]{\ensuremath{\hookleftarrow}},
	breakatwhitespace=false,
	aboveskip={1.5\baselineskip},
  columns=fixed,
  upquote=true,
  extendedchars=true
% frame=single,
% backgroundcolor=\color{lbcolor},
}
\usepackage{color}
%\ifpdf

    \usepackage[pdftex]{graphicx}
    % Update title and author below...
    \usepackage[pdftex,plainpages=false,breaklinks=true,colorlinks=true,urlcolor=blue,citecolor=blue,%
                                       linkcolor=blue,bookmarks=true,bookmarksopen=true,%
                                       bookmarksopenlevel=3,pdfstartview=FitV,
                                       pdfauthor=Christopher Gibson,
                                       pdftitle=Point-Based Color Bleeding With Volumes,
                                       pdfkeywords={thesis, masters, cal poly, volume rendering, global illumination}
                                       ]{hyperref}
    %Options with pdfstartview are FitV, FitB and FitH
    \pdfcompresslevel=1

%\else
%    \usepackage{graphicx}
%\fi

\usepackage{amssymb}
\usepackage{amsmath}
\usepackage[letterpaper]{geometry}
\usepackage[overload]{textcase}



%%%%%\bibliographystyle{abbrv}

\setlength{\parindent}{0.25in} \setlength{\parskip}{6pt}

\geometry{verbose,nohead,tmargin=1.25in,bmargin=1in,lmargin=1.5in,rmargin=1.3in}

\setcounter{tocdepth}{2}


% Different font in captions (single-spaced, bold) ------------
\newcommand{\captionfonts}{\small\bf\ssp}

\makeatletter  % Allow the use of @ in command names
\long\def\@makecaption#1#2{%
  \vskip\abovecaptionskip
  \sbox\@tempboxa{{\captionfonts #1: #2}}%
  \ifdim \wd\@tempboxa >\hsize
    {\captionfonts #1: #2\par}
  \else
    \hbox to\hsize{\hfil\box\@tempboxa\hfil}%
  \fi
  \vskip\belowcaptionskip}
\makeatother   % Cancel the effect of \makeatletter
% ---------------------------------------

\begin{document}

% Declarations for Front Matter

% Update fields below!
\title{FlexRender: A distributed rendering architecture for ray tracing huge
scenes on commodity hardware.}
\author{Robert Edward Somers}
\degreemonth{June} \degreeyear{2012} \degree{Master of Science}
\defensemonth{June} \defenseyear{2012}
\numberofmembers{3} \chair{Zo\"{e} Wood, Ph.D.} \othermemberA{Chris Lupo, Ph.D.} \othermemberB{Phillip Nico, Ph.D.} \field{Computer Science} \campus{San Luis Obispo}
\copyrightyears{seven}



\maketitle

\begin{frontmatter}

% Custom made for Cal Poly (by Mark Barry, modified by Andrew Tsui).
\copyrightpage

% Custom made for Cal Poly (by Andrew Tsui).
\committeemembershippage

\begin{abstract}

As the quest for more realistic computer graphics marches steadily on, the
demand for rich and detailed imagery is greater than ever. Unfortunately, our
appetite for large and complex geometry is quickly outpacing advances in the
hardware used to render it. Scenes with hundreds of millions or even billions
of polygons are not only desired, they are demanded.

Techniques such as normal mapping and level of detail have attempted to address
the problem by reducing the amount of geometry in a scene. This is problematic
for applications that desire or demand access to the scene's full geometric
complexity at render time. More recently, out-of-core techniques have provided
methods for rendering large scenes when the working set is larger than the
available system memory.

We propose a distributed rendering architecture based on message-passing that
is designed to partition scene geometry across a cluster of commodity machines
in a spatially coherent way, allowing the entire scene to remain in-core and
enabling the construction of hierarchical spatial acceleration structures in
parallel. The results of our implementation show over an order of magnitude
speedup in rendering time compared the traditional approach, while keeping
memory overhead for message queuing around 1\%.

\textbf{TODO: Triple check that claim with the Toy Store scene results.}

\end{abstract}

%\begin{acknowledgements}

%   Thank you...

%\end{acknowledgements}


\tableofcontents


\listoftables

\listoffigures

\end{frontmatter}

\pagestyle{plain}




\renewcommand{\baselinestretch}{1.66}


% ------------- Main chapters here --------------------





\chapter{Introduction}
\label{intro}

\textbf{TODO}

\section{Our Contribution}
\label{contribution}

\textbf{TODO}

\chapter{Background}
\label{background}

\textbf{TODO}

\section{Light and Radiometry}
\label{radiometry}

\textbf{TODO}

\section{Ray Tracing}
\label{raytracing}

\textbf{TODO}

\section{Bounding Volume Hierarchies}
\label{bvhs}

\textbf{TODO}

\section{Morton Coding and the Z-Order Curve}
\label{mortoncoding}

\textbf{TODO}

\chapter{Related Work}
\label{relatedwork}

\textbf{TODO}

\chapter{FlexRender Architecture}
\label{architecture}

\textbf{TODO}

\chapter{Results}
\label{results}

\textbf{TODO}

\chapter{Future Work}
\label{futurework}

\textbf{TODO}

\clearpage
\bibliography{pcbex}
\bibliographystyle{plain}
%\addcontentsline{toc}{chapter}{Bibliography}

\section*{Image Results}

\end{document} much faster 
